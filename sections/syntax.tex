\section{Introducing type patterns}

We introduce type patterns from the perspective of introducing a pattern matching mechanism, as this matches more directly with the way we map them into the pre-existing hybrid type system of SaC.
An alternative way to think about type patterns is to see them as ways to introduce types, through the introduction of variables within type signatures of function definitions.

\subsection{Syntax of type patterns}

We start out with the syntax of type patterns, shown in Figure~\ref{fig:syntax}.

\begin{figure}[hbt]
\begin{verbatim}
<pattern>  ::= <type> '[' <feature> (',' <feature>)* ']'

<feature>  ::= <single>
            |  <multiple>

<single>   ::= '.'
            |  <num>
            |  <id>

<multiple> ::= '*'
            |  '+'
            |  <num> ':' <id>
            |  <id> ':' <id>
\end{verbatim}
\caption{EBNF for type patterns}
\label{fig:syntax}
\end{figure}

\noindent
This syntax extends of the notion of array types from SaC, providing more syntactical flexibility and introducing variables within these types.
Similar to the types in SaC, type patterns consist of an element type followed by a shape specification in square brackets.
The shape specification is a pattern, which consists of a list of `features'.
These features either describe a single rank of the argument, or a larger slice (a sub-vector) of its shape vector.

Single dimensions are captured by the \verb|<single>| rule in Figure~\ref{fig:syntax} and can be denoted by the `\tpdot{}' symbol, a constant number, or a variable name.
A fixed number requires the length of the corresponding rank component to match that very number, whereas the `\texttt{.}' symbol allows for an arbitrary extent.
If instead a variable is being used, the extent found needs to match all other occurrences of that very variable within a given function definition.
For example, consider the type pattern \texttt{int[n,n]}.
It allows for square matrices of arbitrary size $\texttt{n} \times \texttt{n}$ where \texttt{n} is a non-negative integer value.

Rank-polymorphism requires the ability to denote a statically unknown number of ranks.
To cater for this, the proposed type patterns support features that capture slices of a given shape.
These are captured by the \verb|<multiple>| rule in Figure~\ref{fig:syntax}.
We distinguish four cases, depending on the rank of the slice we want to match, and whether we are interested in the slice itself or not.
If we are not interested in the slice itself, we can use one of the symbols `\texttt{+}' and `\texttt{*}'.
They match any slice of at least rank 1 or 0, respectively.

If we are interested in the slice itself, the feature consists of two components separated by a `\texttt{:}' symbol.
The first component relates to the rank of that slice, whereas the second component relates to the slice itself, i.e. the lengths of the corresponding axes.
For the rank component, similar to the single-rank features, we can either use a fixed number or an identifier.
For example, \texttt{int[3:shp]} matches rank 3 arrays only, and matches the entire shape vector against the variable \texttt{shp} that will thus be a three-element vector.
Note here that variables after the `\texttt{:}' symbol, such as \texttt{shp}, denote slices and therefore always represent vectors, even if they match no ranks (empty vector) or individual ranks (singleton vector).
For example, a type pattern \texttt{int[1:n,1:n]}, similar to the \texttt{int[n,n]} example above, matches arbitrary sized square matrices, but \texttt{n} now is a singleton vector rather than a scalar value.

Finally we have the case where the rank of the slice is also a variable, enabling a match of a variable-rank slice of an array shape.
Here, the first component matches against the rank of the slice, and the second component again matches against the slice itself.
For example, the pattern \texttt{int[d:shp]} matches arrays of arbitrary shape, and binds the variable \texttt{d} to the rank of the array, and the variable \texttt{shp} to the shape of the array.

These features can be combined to create complex type patterns.
Consider for example a type pattern \texttt{int[5,n,d:shp]} \texttt{v}.
It matches any shape of at least rank 2, whose extent in the first axis is exactly five, and whose extent in the second axis is \texttt{n}.
Any potentially remaining axes are bound to the vector \texttt{shp}.
These variables (\texttt{n}, \texttt{d}, and \texttt{shp}) can also be used in the function body.
Following are some examples for arguments \texttt{v} and the resulting values of \texttt{n}, \texttt{d}, and \texttt{shp}.

\begin{table}[ht!]
\ttfamily
\begin{tabular}{lllll}\hline
    shape(v)             & 5 & n & d & shp       \\\hline
    \text{}[5,1]         & 5 & 1 & 0 & []        \\
    \text{}[5,1,7]       & 5 & 1 & 1 & [7]       \\
    \text{}[5,0,1,2]     & 5 & 0 & 2 & [1,2]     \\
    \text{}[5,2,1,2,3,4] & 5 & 2 & 4 & [1,2,3,4] \\\hline
\end{tabular}
\end{table}

\noindent
Variable names may be used more than once.
For example, the variable \texttt{n} can be reused in a variable-rank feature \texttt{[5,n,n:shp]}~\texttt{v}.
This pattern matches arrays of at least rank 2 whose extent in the second axis determines the number of ranks that follow.
Some examples of matching shapes are:

\begin{table}[ht!]
\ttfamily
\begin{tabular}{llll}\hline
    shape(v)             & 5 & n & shp       \\\hline
    \text{}[5,0]         & 5 & 0 & []        \\
    \text{}[5,1,0]       & 5 & 1 & [0]       \\
    \text{}[5,1,42]      & 5 & 1 & [42]      \\
    \text{}[5,4,1,2,3,4] & 5 & 4 & [1,2,3,4] \\\hline
\end{tabular}
\end{table}

\noindent
Note that \texttt{d} in the first example, and \texttt{n} in the second example, are equal to the length of \texttt{shp}, which is always the case for such variable-rank patterns.
